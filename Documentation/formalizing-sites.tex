\documentclass{article}
\usepackage{amsmath}
\usepackage{fullpage}

\title{Translating Sites into Lean}
\author{Alexandra Aiello, Eric Klavins, Leopold Meyer, James Martin\\\textit{Math AI Lab, University of Washington}}

\begin{document}
\maketitle

\section{Abstract}
We set out to show that Mathlib did it wrong, but ended up showing that Mathlab did it right.
However, we learned a lot along the way, some of which we think would be valuable to you.
Our contributions include a practical and formal explanation and comparison of four
definitions of site and proofs of several examples. This paper might be valuable to
mathematicians wanting to understand the motivations for Mathlib definitions
and the translation of informal mathematics to formal ones.

\section{Introduction}
The Stacks project is widely regarded as the golden standard for algebraic topology/geometry.
The advisors regard Mathlib as ill-documented, building upon dubious definitions
made unilaterally, potentially containing incoherent definitions across branches of mathematics,
and lacking in examples. (TODO: We need to justify the purpose for the project without
such blatant and strong opinions.) Consequently, we set out to formalize sites, sheaves,
and stacks precisely following the Stacks project with an emphasis on examples.
Our project progressively diverged from this goal as we encountered friction between
the language of informal mathematics and the language of type theory.

\section{Definitions}
We progressed through several definitions of Site, which we will introduce and compare below.
According to nLab, all of these sites are equivalent as the underlying site of a sheaf,
but they are not in bijection.

We characterize these sites by the underlying definition of Precover.
The first three definitions share almost exactly the same axioms,
differing only to accomodate the change in definition of precover.
On the other hand, the fourth definition has substantially different axioms.

A site is defined over some category $C$.
It is a set of covers such that isomorphisms are covers, covers are stable under pullback,
and the set of covers is transitive. (TODO: Elaborate)

To contrast, the Grothendieck topology definition (extremely dubious upcoming)
contains the maximal sieve, stability under pullback, and transitivity,
with the additional constraint that the ``precovers'' are actually sieves---
individually closed under precomposition. (TODO: Elaborate, clarify, probably move.)

\subsubsection*{The indexed families definition}
Let $X \in C$. We define an \textit{indexed precover} $U$ over $X$ as an indexing set $I$
and a pair of indexed families $U_i : C$ and $f_i : U_i \to X$. This is a literal
translation of the definition used by the Stacks project.

TODO: Include pretopology axioms. In general, possibly try to include informal
definitions/explanations and Lean definitions/proofs side-by-side?

\subsubsection*{The sets of morphisms definition}
Let $X \in C$. We define a \textit{set precover} $U$ over $X$ as a set of morphisms
$(Y, f) : (Y : C) \times (Y \to X)$.

Every set precover immediately generates an indexed precover where $U$ acts as
its own indexing set. Specifically, we define $U_{(Y,f)} := Y$ and $f_{(Y,f)} := f$.
Conversely, we can easily recover our set precover as $\{ (U_i, f_i) | i \in I \}$.

Although these definitions look very similar, they are not in bijection.
In particular, an indexed precover may contain two $i,j \in I$ such that $i \ne j$
but $f_i = f_j$, i.e. the precover contains the same morphism twice.
Therefore, even disregarding the choice of indexing set, the definitions are not equal.

We regard this duplication as a syntactic artifact that has no bearing on the semantic
notion of precover; informally, we might regard the definitions as equivalent.

However, in Lean, the difference is significant when considering equality of sites---
the redundant information \textit{cannot} be ignored---and the indexing notation
that is convenient for informal mathematics becomes inconvenient. (TODO: Elaborate)
We expect that the set precover would be preferable to the indexed definition for
the purpose of formalizing mathematics in Lean---even when using the Stacks project
as a reference.

\subsubsection*{The presieve definition}
Let $X \in C$. We define a $\textit{presieve}$ $U$ over $X$ as a family of sets of
morphisms $U_Y : Y \to X$. This definition is in Mathlib under the name ``Pretopology''.

Presieves and sets of morphisms are trivially equivalent.
Specifically, we define our set precover as $\{ (Y, f) | Y \in C, f \in U_Y \}$
and our presieve with $U_Y := \{ f | (Y, f) \in U \}$.

Mathlib does not contain our literal definition of ``set precover''
or the corresponding notion of ``site'', but it does have theorems
that translate between the respective precovers and site axioms.
(TODO: Which were they, again?)

This translation reflects the isomorphism between
$(\exists Y, Y \to X) \to Prop$ and $\forall Y, (Y \to X) \to Prop$,
an operation that type theorists call ``currying''.
Although the set precover definition is more intuitive coming from the perspective
of the indexed set definition, the computer scientists on our team intuitively
prefer presieves because $\forall$ is privileged over $\exists$ in most
functional programming languages. Additionally, as the name suggests,
presieves more closely relate to sieves---our next definition of covering.

The fact that Mathlib uses presieves over set precovers likely makes the
point moot; the strong advantage of using Mathlib's existing theorems makes
presieves strongly preferable \textit{a posteriori}.

\subsubsection*{The sieve definition}
TODO: Explain the axioms and go into more detail.

A sieve is a presieve that is closed under precomposition.

A topology is generated by/the completion of a pretopology.
Although both pretopologies and Grothendieck topologies are often called
sites, unlike with the three former definitions, they are not ``morally''
the same (although they are equivalent as the underlying site as a presheaf).

Grothendieck topologies are the \textit{most restrictive} definition.
The algebraic topologists/geometers we consulted claim that the additional
structure of a sieve is not necessary for their work, but the advantage
of the more restrictive definition is that it contains no redundant information,
which is semantically cleaner and makes \textit{equivalent} sites actually \textit{equal}.
Furthermore, since all pretopologies \textit{generate} Grothendieck topologies,
this doesn't actually restrict the expressiveness of our notion of site.
This may make Grothendieck topologies preferable to pretopologies from a formal
perspective, and indeed, this is the primary definition that Mathlib uses.

\section{Examples}

\section{Conclusion}
Our(?) opinions:
\begin{itemize}
\item Explicit indices are the most convenient for informal mathematics, but awkward in Lean.
\item Precovers are the most intuitive translation of precovers into type-theoretic language.
\item Presieves are the most convenient representation to work with in Lean.
\item Grothendieck topologies are the most semantically correct.
\end{itemize}

The most challenging part of formalized mathematics is not in understanding definitions,
but in writing proofs, and presieves and Grothendieck topologies are the most convenient
to use in Lean. It happens that these are precisely the definitions used in Mathlib.

\subsection{Future work}
Our project was originally to formalize the Stacks project as directly as possible,
but we've found that Mathlib's existing definitions are the most suitable for Lean.
Although we could try to translate more definitions, we expect that we would come
to the same conclusions.

If we were to continue translating the Stacks project, our main product will most likely be
documentation for mathematicians learning to translate from informal mathematics to lean
(or the converse for computer scientists).

Alternatively, we could pivot from a lockstep translation of the Stacks project to Lean to
more general work demonstrating examples, proving theorems, and writing documentation---drawing
on the Stacks project for guidance, but not taking it as gospel.

\end{document}